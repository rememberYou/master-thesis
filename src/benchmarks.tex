
\chapter{Benchmarks}
\label{chap:benchmarks}

For these benchmarks, only the maximum number of walks per entity and the
maximum depth per walk is varied. Furthermore, due to a time issue (cf. Section
\ref{sec:objectives:problems}) these benchmarks are performed on \texttt{MUTAG},
a graph of moderate size composed of
\SI{74567}{triples}\footnote{\textbf{SELECT} (COUNT(*) AS ?triples)
\textbf{WHERE} \{ ?s ?p ?o \}} \SI{22534}{entities }\footnote{\textbf{SELECT}
(COUNT(\textbf{DISTINCT} ?s) AS ?entities) \textbf{WHERE} \{ ?s a \}}, and
\SI{24}{relations}\footnote{\textbf{SELECT} (COUNT(\textbf{DISTINCT} ?p) AS
?relations) \textbf{WHERE} \{ ?s ?p ?o \}}. Finally, each value entered in these
benchmarks is the result of the average of five values.

\section{Setup}
\label{sec:setup}

Benchmarks related to embeddings techniques and walking strategies are directly
launched on IDLab's\footnote{Research group of imec.} servers with \SI{4}{CPUs},
\SI{64}{\giga\byte} RAM, and one GPU. Those related to sampling strategies are
launched directly on a ThinkPad machine with \SI{4}{CPUs} and
\SI{16}{\giga\byte} of RAM. This physical device change is made since, except
for \texttt{UniformSampler}, the sampling strategies only work on locally stored
KGs. As the IDLab servers interact with the KGs via SPARQL endpoints, they were
not used for these benchmarks. Finally, the benchmarks use 340 training entities
and attempt to predict 68 test entities, which is a standard for \texttt{MUTAG}.

\section{Results}
\label{sec:results}

This section contains the results of the different embedding techniques, walking
strategies, and sampling strategies for \texttt{MUTAG}.


\subsection{Embedding Techniques}
\label{subsec:embedding:techniques}

FastText, BERT and Word2Vec are trained on the basis of ten epochs. In addition,
FastText and Word2Vec use twenty negative words with a vector size of 500. For
its splitting function, FastText uses a primary splitting function where the
\texttt{\#} symbol splits each entity Finally, each embedding technique uses a
\texttt{RandomWalker} and an \texttt{UniformSampler}.

About the training of BERT, the latter is only trains on
\texttt{MUTAG}. However, in case of multiple KGs, it is important to re-train it
for each different KGs. Similarly, if BERT is trained with too few walks, it
will be necessary to retrain it with a larger number of walks. In this case,
online learning is important to avoid to retrain the whole model which can be
time-consuming. Unline BERT that can take hours, days for training the model,
Word2Vec and FastText take a few minutes to tens of minutes to train, which is
significant difference.

\begin{table}[!ht]
  \centering
  \begin{tabular}{rl}
    \toprule
    \textbf{Hyperparameter} & \textbf{Value} \\
    \midrule
    \textbf{Epochs} & 10 \\
    \textbf{Warmup Steps} & 500 \\
    \textbf{Weight Decay} & 0.01 \\
    \textbf{Learning Rate} & 2e-5 \\
    \textbf{Batch Size} & 16 \\
    \bottomrule
  \end{tabular}
  \caption{Basic Hyperparameters Used for Training the BERT Model.}
  \label{tab:bert:hyperparameters}
\end{table}

In Table \ref{tab:bert:hyperparameters}, The values of these basic
hyperparameters were chosen after several tests using a Grid Search with Cross
Validation and depending on the training time. More precisely, a training of the
BERT model with these hyperparameters with MUTAG and the same hardware
characteristics as those of the IDLab servers, is done between 25 minutes and
few hours.

\begin{table}[!ht]
  \centering
  \resizebox{\textwidth}{!}{%
    \begin{tabular}{cccS[table-format=2.2]@{${}\pm{}$}S[table-format=1.1]c}
      \toprule
      \textbf{Embedding Technique} & \textbf{Max. Depth} & \textbf{Max. Walks}
      & \multicolumn{2}{c}{\textbf{Accuracy} (\SI{}{\percent})} & \textbf{Rank} \\
      \midrule
      \texttt{FastText(negative=20,vector\_size=500)} & \multirow{3}{*}{2} & \multirow{3}{*}{250} & 79.71 & 2.35 & 1 \\
      \texttt{Word2Vec(negative=20,vector\_size=500)} & & & 76.76 & 1.71 & 2 \\
      \texttt{BERT(learning\_rate=2e-5,batch\_size=16)} & & & 70.59 & 5.88 & 3 \\
      \midrule
      \texttt{FastText(negative=20,vector\_size=500)} & \multirow{3}{*}{4} & \multirow{3}{*}{250} & 77.06 & 1.50 & 1 \\
      \texttt{Word2Vec(negative=20,vector\_size=500)} & & & 75.00 & 1.61 & 2 \\
      \texttt{BERT(learning\_rate=2e-5,batch\_size=16)} & & & 74.26 & 2.21 & 3 \\
      \midrule
      \texttt{FastText(negative=20,vector\_size=500)} & \multirow{3}{*}{6} & \multirow{3}{*}{250} & 82.35 & 1.86 & 1 \\
      \texttt{BERT(learning\_rate=2e-5,batch\_size=16)} & & & 76.32 & 3.24 & 2 \\
      \texttt{Word2Vec(negative=20,vector\_size=500)} & & & 74.71 & 2.35 & 3 \\
      \bottomrule
    \end{tabular}
  }%
  \caption{Evaluation of the Embedding Techniques for \texttt{MUTAG} According
    to the Maximum Depth per Walk.}
  \label{benchmarks:embedders:mutag:depth}
\end{table}

In Table \ref{benchmarks:embedders:mutag:depth}, regardless of the maximum depth
per walk chosen for the same number of walks per entity, FastText indicates a
model's accuracy above Word2Vec. Specifically, FastText allows an average
increase of the model's accuracy of 4.22 times the one given by Word2Vec. In
addition, FastText provides an excellent model's accuracy with \texttt{MUTAG}
for a maximum depth per walk of 6. For BERT, the latter shows better results for
larger maximum depth per walk.

\begin{figure}[!ht]
  \centering
  \resizebox{\textwidth}{!}{%
  \begin{tikzpicture}
    \begin{axis}[
      scale only axis,
      grid=major,
      grid style={dashed,gray!30},
      height=6cm,
      width=9cm,
      legend cell align={left},
      legend entries={
        \footnotesize{\texttt{BERT(learning\_rate=2e-5,batch\_size=16)}},
        \footnotesize{\texttt{Word2Vec(negative=20,vector\_size=500)}},
        \footnotesize{\texttt{FastText(negative=20,vector\_size=500)}}
      },
      legend style={
        legend pos=outer north east,
        font=\small
      },
      ylabel={Accuracy},
      xlabel={Maximum Depth per Walk},
      xtick={2,4,6},
      ytick={75,77,79.70,82.30},
      ]

      \addplot[red,mark=*,error bars/.cd, y dir=both, y explicit]
      table[x=max_depth,y=accuracy,col sep=comma] {data/embedders/max-depth/bert.csv};
      \addplot[blue,mark=*,error bars/.cd, y dir=both, y explicit]
      table[x=max_depth,y=accuracy,col sep=comma] {data/embedders/max-depth/word2vec.csv};
      \addplot[green,mark=*,error bars/.cd, y dir=both, y explicit]
      table[x=max_depth,y=accuracy,col sep=comma] {data/embedders/max-depth/fasttext.csv};
    \end{axis}
  \end{tikzpicture}
  }%
  \caption{Evaluation of the Embedding Techniques for \texttt{MUTAG} According
    to the Maximum Depth per Walk.}
  \label{fig:benchmarks:embedders:depth}
\end{figure}

In Figure \ref{fig:benchmarks:embedders:depth}, the curves of Word2Vec and
FastText have an almost identical trajectory, except for a maximum depth per
walk of 6. In this case, the accuracy model of FastText increases, while the
accuracy model of Word2Vec decreases. Finally, BERT's accuracy is proportional
to the maximum depth per walk. As well as the time needed to train the model.

\begin{table}[!ht]
  \centering
  \resizebox{\textwidth}{!}{%
    \begin{tabular}{cccS[table-format=2.2]@{${}\pm{}$}S[table-format=1.2]c}
      \toprule
      \textbf{Embedding Technique} & \textbf{Max. Depth} & \textbf{Max. Walks}
      & \multicolumn{2}{c}{\textbf{Accuracy} (\SI{}{\percent})} & \textbf{Rank} \\
      \midrule
      \texttt{FastText(negative=20,vector\_size=500)} & \multirow{3}{*}{4} & \multirow{3}{*}{100} & 77.94 & 1.61 & 1 \\
      \texttt{Word2Vec(negative=20,vector\_size=500)} & & & 71.47 & 2.20 & 2 \\
      \texttt{BERT(learning\_rate=2e-5,batch\_size=16)} & & & 69.43 & 1.73 & 3 \\
      \midrule
      \texttt{FastText(negative=20,vector\_size=500)} & \multirow{3}{*}{4} & \multirow{3}{*}{250} & 77.35 & 3.90 & 1 \\
      \texttt{Word2Vec(negative=20,vector\_size=500)} & & & 74.71 & 2.53 & 2 \\
      \texttt{BERT(learning\_rate=2e-5,batch\_size=16)} & & & 73.54 & 2.28 & 3 \\
      \midrule
      \texttt{FastText(negative=20,vector\_size=500)} & \multirow{3}{*}{4} & \multirow{3}{*}{500} & 76.18 & 1.71 & 1 \\
      \texttt{BERT(learning\_rate=2e-5,batch\_size=16)} & & & 75.24 & 2.37 & 2 \\
      \texttt{Word2Vec(negative=20,vector\_size=500)} & & & 73.53 & 1.86 & 3 \\
      \midrule
      \texttt{FastText(negative=20,vector\_size=500)} & \multirow{3}{*}{4} & \multirow{3}{*}{1000} & 77.35 & 2.73 & 1 \\
      \texttt{BERT(learning\_rate=2e-5,batch\_size=16)} & & & 76.58 & 1.17 & 2 \\
      \texttt{Word2Vec(negative=20,vector\_size=500)} & & & 74.41 & 3.03 & 3 \\
      \bottomrule
    \end{tabular}
  }%
  \caption{Evaluation of the Embedding Techniques for \texttt{MUTAG} According
    to the Maximum Number of Walks per Entity}
  \label{benchmarks:embedders:mutag:walks}

\end{table}

In Table \ref{benchmarks:embedders:mutag:walks}, regardless of the number of
walks chosen for the same maximum depth per walk, FastText indicates a model's
accuracy above Word2Vec. Specifically, FastText allows an average increase of
the model's accuracy of 3.675 times the one given by Word2Vec. For BERT, the
latter shows better results for larger maximum depth per walk, but performs less
well for smaller maximum depth per walk. For BERT, the latter indicates an
interesting model's accuracy for 500 and 1000 walks. However, the results are
not as exceptional for a lower maximum of walks per entity.

\begin{figure}[!ht]
  \centering
  \resizebox{\textwidth}{!}{%
    \begin{tikzpicture}
      \begin{axis}[
        scale only axis,
        grid=major,
        grid style={dashed,gray!30},
        height=6cm,
        width=9cm,
        legend cell align={left},
        legend entries={
          \footnotesize{\texttt{BERT(learning\_rate=2e-5,batch\_size=16)}},
          \footnotesize{\texttt{Word2Vec(negative=20,vector\_size=500)}},
          \footnotesize{\texttt{FastText(negative=20,vector\_size=500)}}
        },
        legend style={
          legend pos=outer north east,
          font=\small
        },
        ylabel={Accuracy},
        xlabel={Maximum Number of Walks per Entity},
        xtick={100,250,500,1000},
        ytick={71.50,73.50,74.50,76.20,78}
        ]

        \addplot[red,mark=*,error bars/.cd, y dir=both, y explicit]
        table[x=max_walk,y=accuracy,col sep=comma] {data/embedders/max-walks/bert.csv};
        \addplot[blue,mark=*,error bars/.cd, y dir=both, y explicit]
        table[x=max_walk,y=accuracy,col sep=comma] {data/embedders/max-walks/word2vec.csv};
        \addplot[green,mark=*,error bars/.cd, y dir=both, y explicit]
        table[x=max_walk,y=accuracy,col sep=comma] {data/embedders/max-walks/fasttext.csv};
      \end{axis}
    \end{tikzpicture}
  }%
  \caption{Evaluation of the Embedding Techniques for \texttt{MUTAG} According
    to the Maximum Number of Walks per Entity.}
  \label{fig:benchmarks:embedders:walks}
\end{figure}

In Figure \ref{fig:benchmarks:embedders:walks}, the curves of Word2Vec and
FastText still have an almost identical trajectory, except for a maximum number
of walks per entity of 250. In this case, the accuracy model of Word2Vec
increases, while the accuracy model of FastText decreases. Finally, BERT's
accuracy is also proportional to the maximum number of walks per entity. As well
as the time needed to train the model.

\newpage

\begin{table}[!ht]
  \centering
  \begin{tabular}{lc}
    \toprule
    \textbf{Embedding Technique} & \textbf{Average Rank} \\
    \midrule
    \texttt{FastText(negative=20,vector\_size=500)} & 1 \\
    \texttt{Word2Vec(negative=20,vector\_size=500)} & 2 \\
    \texttt{BERT(learning\_rate=2e-5,batch\_size=16)} & 3 \\
    \bottomrule
  \end{tabular}
  \caption{Evaluation of the Average Rank of the Embedding Techniques for \texttt{MUTAG}.}
  \label{tab:benchmark:embedders:average:rank}
\end{table}

In Figure \ref{tab:benchmark:embedders:average:rank}, Word2Vec is the winning
embedding techniques in these benchmarks for \texttt{MUTAG}, followed by
FastText, and BERT.

%%% Local Variables:
%%% mode: latex
%%% TeX-master: "../../master-thesis"
%%% End:


\subsection{Walking Strategies}
\label{subsec:walkers}

Each walking strategy is evaluated using \texttt{UniformSampler} as sampling strategy and
Word2Vec as embedding technique. For these benchmarks, Word2Vec keeps the same
hyperparameters as given in Section \ref{subsec:embedding:techniques}, namely
ten epochs, twenty negative words and a vector size of 500.

\begin{table}[!ht]
  \centering
  \resizebox{\textwidth}{!}{%
    \begin{tabular}{lccS[table-format=2.2]@{${}\pm{}$}S[table-format=1.2]c}
      \toprule
      \textbf{Walker} & \textbf{Max. Depth} & \textbf{Max. Walks}
      & \multicolumn{2}{c}{\textbf{Accuracy} (\SI{}{\percent})} & \textbf{Rank} \\
      \midrule
      \texttt{RandomWalker} & \multirow{6}{*}{2} & \multirow{6}{*}{250} & \textbf{77.94} & 2.08 & 1 \\
      \texttt{NGramWalker(grams=3)} & & & 76.47 & 1.32 & 2 \\
      \texttt{HALKWalker(freq\_threshold=0.01)} & & & 75.59 & 2.39 & 3 \\
      \texttt{SplitWalker} & & & 74.71 & 2.53 & 4 \\
      \texttt{WalkletWalker} & & & 72.06 & 1.32 & 5 \\
      \texttt{AnonymousWalker} & & & 65.29 & 1.76 & 6 \\
      \midrule
      \texttt{HALKWalker(freq\_threshold=0.01)} & \multirow{6}{*}{4} & \multirow{6}{*}{250} & \textbf{78.82} & 1.50 & 1 \\
      \texttt{SplitWalker} & & & 77.35 & 4.01 & 2 \\
      \texttt{RandomWalker} & & & 76.76 & 6.06 & 3 \\
      \texttt{NGramWalker(grams=3)} & & & 75.88 & 3.90 & 4 \\
      \texttt{WalkletWalker} & & & 73.82 & 1.95 & 5 \\
      \texttt{AnonymousWalker} & & & 66.47 & 1.44 & 6 \\
      \midrule
      \texttt{HALKWalker(freq\_threshold=0.01)} & \multirow{6}{*}{6} & \multirow{6}{*}{250} & \textbf{81.18} & 4.87 & 1 \\
      \texttt{SplitWalker} & & & 79.71 & 2.16 & 2 \\
      \texttt{NGramWalker(grams=3)} & & & 77.65 & 1.95 & 3 \\
      \texttt{RandomWalker} & & & 75.29 & 2.16 & 4 \\
      \texttt{WalkletWalker} & & & 71.76 & 1.10 & 5 \\
      \texttt{AnonymousWalker} & & & 67.65 & 1.86 & 6 \\
      \bottomrule
    \end{tabular}
    }%
  \caption{Evaluation of the Accuracy of Walking Strategies for \texttt{MUTAG} According
    to the Maximum Depth per Walk.}
  \label{benchmarks:walkers:mutag:depth}
\end{table}

In Table \ref{benchmarks:walkers:mutag:depth}, regardless of the maximum depth
per walk chosen for the same number of walks per entity, \texttt{HALKWalker}
indicates a model's accuracy above \texttt{SplitWalker}. However, these two
walking strategies provide better results for more significant maximum depths per
walk. While \texttt{RandomWalker} indicates a good model's accuracy for small maximum
depth per walk.

\newpage

\begin{figure}[!ht]
  \centering
  \resizebox{\textwidth}{!}{%
  \begin{tikzpicture}
    \begin{axis}[
      scale only axis,
      grid=major,
      grid style={dashed,gray!30},
      height=6cm,
      width=9cm,
      legend cell align={left},
      legend entries={
        \texttt{AnonymousWalker},
        \texttt{HALKWalker(freq\_threshold=0.01)},
        \texttt{NGramWalker(grams=3)},
        \texttt{RandomWalker},
        \texttt{SplitWalker},
        \texttt{WalkletWalker},
      },
      legend style={
        legend pos=outer north east,
        font=\small
      },
      ylabel={Accuracy},
      xlabel={Maximum Depth per Walk},
      xtick={2, 4, 6},
      ytick={65.30,67,71.80,73.80,76,78,79.70,81.20},
      ]

      \addplot[brown,mark=*,error bars/.cd, y dir=both, y explicit]
      table[x=max_depth,y=accuracy,col sep=comma] {data/walkers/max-depth/anonymous.csv};
      \addplot[yellow,mark=*,error bars/.cd, y dir=both, y explicit]
      table[x=max_depth,y=accuracy,col sep=comma] {data/walkers/max-depth/halk.csv};
      \addplot[blue,mark=*,error bars/.cd, y dir=both, y explicit]
      table[x=max_depth,y=accuracy,col sep=comma] {data/walkers/max-depth/ngram.csv};
      \addplot[green,mark=*,error bars/.cd, y dir=both, y explicit]
      table[x=max_depth,y=accuracy,col sep=comma] {data/walkers/max-depth/random.csv};
      \addplot[red,mark=*,error bars/.cd, y dir=both, y explicit]
      table[x=max_depth,y=accuracy,col sep=comma] {data/walkers/max-depth/split.csv};
      \addplot[darkPurple,mark=*,error bars/.cd, y dir=both, y explicit]
      table[x=max_depth,y=accuracy,col sep=comma] {data/walkers/max-depth/walklet.csv};
    \end{axis}
  \end{tikzpicture}
  }%
  \caption{Evaluation of the Accuracy of Walking Strategies for MUTAG According to the Maximum Depth per Walk.}
  \label{fig:benchmarks:walkers:depth}
\end{figure}

In Figure \ref{fig:benchmarks:walkers:depth}, the curve of \texttt{SplitWalker}
and \texttt{HALKWalker} have an identical trajectory, except that
\texttt{HALKWalker} indicates better model's accuracy than
\texttt{SplitWalker}. In addition, they return better model's accuracy than the
other walking strategies.

\begin{table}[!ht]
  \centering
  \resizebox{\textwidth}{!}{%
    \begin{tabular}{lccS[table-format=2.2]@{${}\pm{}$}S[table-format=1.2]c}
      \toprule
      \textbf{Walker} & \textbf{Max. Depth} & \textbf{Max. Walks}
      & \multicolumn{2}{c}{\textbf{Accuracy} (\SI{}{\percent})} & \textbf{Rank} \\
      \midrule
      \texttt{SplitWalker} & \multirow{6}{*}{4} & \multirow{6}{*}{100} & \textbf{79.12} & 4.20 & 1 \\
      \texttt{HALKWalker(freq\_threshold=0.01)} & & & 77.65 & 2.53 & 2 \\
      \texttt{WalkletWalker} & & & 73.82 & 1.95 & 3 \\
      \texttt{RandomWalker} & & & 72.35 & 3.77 & 4 \\
      \texttt{NGramWalker(grams=3)} & & & 68.82 & 3.99 & 5 \\
      \texttt{AnonymousWalker} & & & 65.59 & 1.18 & 6 \\
      \midrule
      \texttt{SplitWalker} & \multirow{6}{*}{4} & \multirow{6}{*}{250} & \textbf{77.35} & 2.56 & 1 \\
      \texttt{HALKWalker(freq\_threshold=0.01)} & & & 76.76 & 3.65 & 2 \\
      \texttt{RandomWalker} & & & 76.18 & 1.71 & 3 \\
      \texttt{NGramWalker(grams=3)} & & & 73.24 & 3.40 & 4 \\
      \texttt{WalkletWalker} & & & 71.76 & 1.10 & 5 \\
      \texttt{AnonymousWalker} & & & 66.76 & 1.18 & 6 \\
      \midrule
      \texttt{HALKWalker(freq\_threshold=0.01)} & \multirow{6}{*}{4} & \multirow{6}{*}{500} & \textbf{79.12} & 3.99 & 1 \\
      \texttt{SplitWalker} & & & 77.35 & 1.50 & 2 \\
      \texttt{WalkletWalker} & & & 77.06 & 3.55 & 3 \\
      \texttt{RandomWalker} & & & 72.06 & 1.32 & 4 \\
      \texttt{NGramWalker(grams=3)} & & & 71.76 & 1.10 & 5 \\
      \texttt{AnonymousWalker} & & & 65.80 & 1.44 & 6 \\
      \bottomrule
    \end{tabular}
    }%
  \caption{Evaluation of the Accuracy of Walking Strategies for \texttt{MUTAG} According
    to the Maximum Number of Walks per Entity}
  \label{tab:benchmarks:walkers:mutag:walks}
\end{table}

In Table \ref{tab:benchmarks:walkers:mutag:walks}, regardless of the number of
walks chosen for the same maximum depth per walk, \texttt{SplitWalker} indicates
a correct model's accuracy above the average of walking
strategies. Specifically, \texttt{SplitWalker} allows an average model's
accuracy of \SI{78.53}{\percent}.

\newpage

\begin{figure}[!ht]
  \centering
  \resizebox{\textwidth}{!}{%
  \begin{tikzpicture}
    \begin{axis}[
      scale only axis,
      grid=major,
      grid style={dashed,gray!30},
      height=6cm,
      width=9cm,
      legend cell align={left},
      legend entries={
        \texttt{AnonymousWalker},
        \texttt{HALKWalker(freq\_threshold=0.01)},
        \texttt{NGramWalker(grams=3)},
        \texttt{RandomWalker},
        \texttt{SplitWalker},
        \texttt{WalkletWalker},
      },
      legend style={
        legend pos=outer north east,
        font=\small
      },
      ylabel={Accuracy},
      xlabel={Maximum Number of Walks per Entity},
      xtick={100,250,500},
      ytick={65.80,68.80,72,73.20,76,77.50,79.10},
      ]

      \addplot[brown,mark=*,error bars/.cd, y dir=both, y explicit]
      table[x=max_walk,y=accuracy,col sep=comma] {data/walkers/max-walks/anonymous.csv};
      \addplot[yellow,mark=*,error bars/.cd, y dir=both, y explicit]
      table[x=max_walk,y=accuracy,col sep=comma] {data/walkers/max-walks/halk.csv};
      \addplot[blue,mark=*,error bars/.cd, y dir=both, y explicit]
      table[x=max_walk,y=accuracy,col sep=comma] {data/walkers/max-walks/ngram.csv};
      \addplot[green,mark=*,error bars/.cd, y dir=both, y explicit]
      table[x=max_walk,y=accuracy,col sep=comma] {data/walkers/max-walks/random.csv};
      \addplot[red,mark=*,error bars/.cd, y dir=both, y explicit]
      table[x=max_walk,y=accuracy,col sep=comma] {data/walkers/max-walks/split.csv};
      \addplot[darkPurple,mark=*,error bars/.cd, y dir=both, y explicit]
      table[x=max_walk,y=accuracy,col sep=comma] {data/walkers/max-walks/walklet.csv};
    \end{axis}
  \end{tikzpicture}
  }%
  \caption{Evaluation of the Walking Strategies for Different Data Sets
    According to the Maximum Number of Walks per Entity.}
  \label{fig:benchmarks:walkers:walks}
\end{figure}

In Figure \ref{fig:benchmarks:walkers:walks}, the model's accuracy with
\texttt{SplitWalker} tends to stay around \SI{77}{\percent} after a maximum
number of walks of 250. In addition, \texttt{AnonymousWalker},
\texttt{NGramWalker}, and \texttt{RandomWalker} follow the same curve
trajectory. Specifically, they have a peak of accuracy at a maximum number of
walks per entity of 250. In contrast, the other walking strategies have a
decrease of model's accuracy at this stage.

\begin{table}[!ht]
  \centering
  \begin{tabular}{lc}
    \toprule
    \textbf{Walker} & \textbf{Average Rank} \\
    \midrule
    \texttt{HALKWalker(freq\_threshold=0.01)} & 1 \\
    \texttt{SplitWalker} & 2 \\
    \texttt{RandomWalker} & 3 \\
    \texttt{NGramWalker(grams=3)} & 4 \\
    \texttt{WalkletWalker} & 5 \\
    \texttt{AnonymousWalker} & 6 \\
    \bottomrule
  \end{tabular}
  \caption{Evaluation of the Average Rank of the Walking Strategies.}
  \label{tab:benchmark:walkers:average:rank}
\end{table}

In Figure \ref{tab:benchmark:walkers:average:rank},
\texttt{HALKWalker(freq\_threshold=0.01)} is the winning walking strategy in
these benchmarks for \texttt{MUTAG}, followed by \texttt{SplitWalker}, and
\texttt{RandomWalker}.

%%% Local Variables:
%%% mode: latex
%%% TeX-master: "../../master-thesis"
%%% End:


\subsection{Sampling Strategies}
\label{subsec:samplers}

To determine the accuracy impact of \texttt{WideSampler}, the latter is compared
to other sampling strategies in a first time using \texttt{MUTAG} where only the
number of walks and depth are variable. The number of standard entities for
\texttt{MUTAG} being fixed at 340 trained entities, 68 of these serve as testing
entities. As RDF2Vec is unsupervised, including the testing entities in the
training set is not an issue.

\newpage

\begin{table}[!ht]
  \centering
  \resizebox{\textwidth}{!}{%
    \begin{tabular}{lcc S[table-format=2.2]@{${}\pm{}$}S[table-format=1.2]c}
      \toprule
      \textbf{Sampler} & \textbf{Max. Depth} & \textbf{Max. Walks}
      & \multicolumn{2}{c}{\textbf{Accuracy} (\SI{}{\percent})} & \textbf{Rank} \\
      \midrule
      \texttt{ObjPredFreqSampler} & \multirow{13}{*}{2} & \multirow{13}{*}{100} & \textbf{78.82} & 3.17 & 1 \\
      \texttt{ObjFreqSampler} & & & 77.94 & 3.35 & 2 \\
      \texttt{PredFreqSampler(inverse=True)} & & & 77.65 & 2.35 & 3 \\
      \texttt{WideSampler} & & & 76.76 & 1.95 & 4 \\
      \texttt{ObjPredFreqSampler(inverse=True)} & & & 76.76 & 2.16 & 5 \\
      \texttt{PredFreqSampler} & & & 76.76 & 4.87 & 6 \\
      \texttt{PageRankSampler(inverse=True,split=True,alpha=0.85)} & & & 76.18 & 2.16 & 7 \\
      \texttt{PageRankSampler(alpha=0.85)} & & & 76.47 & 2.08 & 8 \\
      \texttt{UniformSampler} & & & 76.18 & 4.50 & 9 \\
      \texttt{ObjFreqSampler(inverse=True,split=True)} & & & 75.88 & 4.01 & 10 \\
      \texttt{PageRankSampler(split=True,alpha=0.85)} & & & 74.71 & 1.71 & 11 \\
      \texttt{PageRankSampler(inverse=True,alpha=0.85)} & & & 74.41 & 1.50 & 12 \\
      \texttt{ObjFreqSampler(inverse=True)} & & & 73.53 & 2.79 & 13 \\
      \midrule
      \texttt{WideSampler} & \multirow{13}{*}{2} & \multirow{13}{*}{250} & \textbf{77.35} & 1.99 & 1 \\
      \texttt{PredFreqSampler} & & & 76.18 & 3.14 & 2 \\
      \texttt{PageRankSampler(split=True,alpha=0.85)} & & & 75.88 & 1.50 & 3 \\
      \texttt{ObjFreqSampler(inverse=True,split=True)} & & & 75.88 & 3.03 & 4 \\
      \texttt{PageRankSampler(inverse=True,split=True,alpha=0.85)} & & & 75.88 & 5.06 & 5 \\
      \texttt{ObjFreqSampler(inverse=true)} & & & 75.59 & 1.99 & 6 \\
      \texttt{ObjFreqSampler} & & & 75.59 & 4.71 & 7 \\
      \texttt{ObjPredFreqSampler(inverse=True)} & & & 75.29 & 1.10 & 8 \\
      \texttt{UniformSampler} & & & 75.29 & 1.44 & 9 \\
      \texttt{ObjPredFreqSampler} & & & 75.29 & 2.35 & 10 \\
      \texttt{PageRankSampler(inverse=True,alpha=0.85)} & & & 75.00 & 2.46 & 11 \\
      \texttt{PageRankSampler(alpha=0.85)} & & & 74.12 & 1.99 & 12 \\
      \texttt{PredFreqSampler(inverse=True)} & & & 73.82 & 2.53 & 13 \\
      \midrule
      \texttt{ObjFreqSampler} & \multirow{13}{*}{2} & \multirow{13}{*}{500} & \textbf{75.29} & 2.16 & 1 \\
      \texttt{PageRankSampler(alpha=0.85)} & & & 75.00 & 0.00 & 2 \\
      \texttt{WideSampler} & & & 74.71 & 1.10 & 3 \\
      \texttt{ObjFreqSampler(inverse=True,split=True)} & & & 74.71 & 3.14 & 4 \\
      \texttt{PageRankSampler(inverse=True,alpha=0.85)} & & & 74.12 & 2.56 & 5 \\
      \texttt{ObjPredFreqSampler(inverse=True)} & & & 73.82 & 2.53 & 6 \\
      \texttt{PageRankSampler(inverse=True,split=True,alpha=0.85)} & & & 73.82 & 2.16 & 7 \\
      \texttt{PageRankSampler(split=True,alpha=0.85)} & & & 73.82 & 2.53 & 8 \\
      \texttt{ObjPredFreqSampler} & & & 73.53 & 1.32 & 9 \\
      \texttt{UniformSampler} & & & 73.53 & 2.63 & 10 \\
      \texttt{PredFreqSampler} & & & 72.65 & 2.73 & 11 \\
      \texttt{ObjFreqSampler(inverse=True)} & & & 72.65 & 2.73 & 12 \\
      \texttt{PredFreqSampler(inverse=True)} & & & 72.35 & 1.95 & 13 \\
      \bottomrule
    \end{tabular}
  }%
  \caption{Accuracy of Sampling Strategies for \texttt{MUTAG} (Part I).}
  \label{tab:benchmarks:samplers:part:1}
\end{table}

In Table \ref{tab:benchmarks:samplers:part:1}, for a maximum depth of walk of 2 with
a maximum number of walks of 250, \texttt{WideSampler} indicates the best
model's accuracy. In addition, the latter gives good precision models for a
maximum number of walks of 100 and 500.

 \begin{figure}[!ht]
  \centering
  \resizebox{0.9\textwidth}{!}{%
    \begin{tikzpicture}
      \begin{axis}[
        scale only axis,
        grid=major,
        grid style={dashed,gray!30},
        height=6cm,
        width=9cm,
        legend cell align={left},
        legend entries={
          \texttt{ObjFreqSampler(inverse=True,split=True)},
          \texttt{ObjFreqSampler(inverse=True)},
          \texttt{ObjFreqSampler},
          \texttt{ObjPredFreqSampler(inverse=True)},
          \texttt{ObjPredFreqSampler},
          \texttt{PageRankSampler(inverse=True,split=True,alpha=0.85)},
          \texttt{PageRankSampler(inverse=True,alpha=0.85)},
          \texttt{PageRankSampler(split=True,alpha=0.85)},
          \texttt{PageRankSampler(alpha=0.85)},
          \texttt{PredFreqSampler(inverse=True)},
          \texttt{PredFreqSampler},
          \texttt{UniformSampler},
          \texttt{WideSampler},
        },
        legend style={
          legend pos=outer north east,
          font=\small
        },
        ylabel={Accuracy},
        xlabel={Maximum Number of Walks per Entity},
        xtick={100,250,500},
        ytick={73,74,76,78},
        ]

        \addplot[medimumBlue,mark=diamond,error bars/.cd, y dir=both, y explicit]
        table[x=max_walk,y=accuracy,col sep=comma] {data/samplers/max-depth/2/objfreq-inv-split.csv};
        \addplot[darkBlue,mark=square,error bars/.cd, y dir=both, y explicit]
        table[x=max_walk,y=accuracy,col sep=comma] {data/samplers/max-depth/2/objfreq-inv.csv};
        \addplot[blue,mark=*,error bars/.cd, y dir=both, y explicit]
        table[x=max_walk,y=accuracy,col sep=comma] {data/samplers/max-depth/2/objfreq.csv};

        \addplot[darkRed,mark=square,error bars/.cd, y dir=both, y explicit]
        table[x=max_walk,y=accuracy,col sep=comma] {data/samplers/max-depth/2/objpredfreq-inv.csv};
        \addplot[red,mark=*,error bars/.cd, y dir=both, y explicit]
        table[x=max_walk,y=accuracy,col sep=comma] {data/samplers/max-depth/2/objpredfreq.csv};

        \addplot[mediumGreen,mark=diamond,error bars/.cd, y dir=both, y explicit]
        table[x=max_walk,y=accuracy,col sep=comma] {data/samplers/max-depth/2/pagerank-inv-split.csv};
        \addplot[darkGreen,mark=square,error bars/.cd, y dir=both, y explicit]
        table[x=max_walk,y=accuracy,col sep=comma] {data/samplers/max-depth/2/pagerank-inv.csv};
        \addplot[yellow,mark=triangle,error bars/.cd, y dir=both, y explicit]
        table[x=max_walk,y=accuracy,col sep=comma]{data/samplers/max-depth/2/pagerank-split.csv};
        \addplot[green,mark=*,error bars/.cd, y dir=both, y explicit]
        table[x=max_walk,y=accuracy,col sep=comma] {data/samplers/max-depth/2/pagerank.csv};

        \addplot[darkPurple,mark=square,error bars/.cd, y dir=both, y explicit]
        table[x=max_walk,y=accuracy,col sep=comma] {data/samplers/max-depth/2/predfreq-inv.csv};
        \addplot[purple,mark=*,error bars/.cd, y dir=both, y explicit]
        table[x=max_walk,y=accuracy,col sep=comma] {data/samplers/max-depth/2/predfreq.csv};

        \addplot[brown,mark=*,error bars/.cd, y dir=both, y explicit]
        table[x=max_walk,y=accuracy,col sep=comma] {data/samplers/max-depth/2/uniform.csv};
        \addplot[black,mark=*,error bars/.cd, y dir=both, y explicit]
        table[x=max_walk,y=accuracy,col sep=comma] {data/samplers/max-depth/2/widesampler.csv};
      \end{axis}
    \end{tikzpicture}
  }%
  \caption{Sampling Strategies for \texttt{MUTAG} According
    to a Maximum Depth per Walk of 2.}
  \label{fig:benchmarks:samplers:part:1}
\end{figure}

In Figure \ref{fig:benchmarks:samplers:part:1}, the model's accuracy for the
different walking strategies is inversely proportional to the maximum number of
walks per entity. Moreover, \texttt{WideSampler} and \texttt{PredFreqSampler}
have two almost similar curves by their trajectory.

\begin{table}[!ht]
  \centering
  \resizebox{0.94\textwidth}{!}{%
    \begin{tabular}{lccS[table-format=2.2]@{${}\pm{}$}S[table-format=1.2]c}
      \toprule
      \textbf{Sampler} & \textbf{Max. Depth} & \textbf{Max. Walks}
      & \multicolumn{2}{c}{\textbf{Accuracy} (\SI{}{\percent})} & \textbf{Rank} \\
      \midrule
      \texttt{WideSampler} & \multirow{13}{*}{4} & \multirow{13}{*}{100} & \textbf{78.82} & 3.03 & 1 \\
      \texttt{PredFreqSampler(inverse=True)} & & & 78.24 & 2.53 & 2 \\
      \texttt{ObjPredFreqSampler} & & & 75.29 & 3.99 & 3 \\
      \texttt{PageRankSampler(inverse=True,alpha=0.85)} & & & 75.00 & 2.63 & 4 \\
      \texttt{PageRankSampler(split=True,alpha=0.85)} & & & 74.12 & 3.79 & 5 \\
      \texttt{ObjFreqSampler(inverse=True)} & & & 73.82 & 2.16 & 6 \\
      \texttt{PredFreqSampler} & & & 73.82 & 2.53 & 7 \\
      \texttt{UniformSampler} & & & 73.24 & 1.95 & 8 \\
      \texttt{ObjPredFreqSampler(inverse=True)} & & & 72.94 & 1.50 & 9 \\
      \texttt{PageRankSampler(inverse=True,split=True,alpha=0.85)} & & & 72.65 & 1.18 & 10 \\
      \texttt{ObjFreqSampler(inverse=True,split=True)} & & & 72.65 & 3.79 & 11 \\
      \texttt{PageRankSampler(alpha=0.85)} & & & 72.65 & 4.22 & 12 \\
      \texttt{ObjFreqSampler} & & & 69.12 & 5.73 & 13 \\
      \midrule
      \texttt{ObjPredFreqSampler(inverse=True)} & \multirow{13}{*}{4} & \multirow{13}{*}{250} & \textbf{79.41} & 2.63 & 1 \\
      \texttt{ObjFreqSampler} & & & 78.53 & 2.56 & 2 \\
      \texttt{PageRankSampler(inverse=True,alpha=0.85)} & & & 77.06 & 2.56 & 3 \\
      \texttt{ObjFreqSampler(inverse=True)} & & & 76.18 & 3.14 & 4 \\
      \texttt{PageRankSampler(inverse=True,split=True,alpha=0.85)} & & & 75.59 & 2.88 & 5 \\
      \texttt{ObjFreqSampler(inverse=True,split=True)} & & & 75.59 & 1.76 & 6 \\
      \texttt{WideSampler} & & & 75.00 & 2.94 & 7 \\
      \texttt{PredFreqSampler(inverse=True)} & & & 74.71 & 2.85 & 8 \\
      \texttt{PredFreqSampler} & & & 74.41 & 2.73 & 9 \\
      \texttt{UniformSampler} & & & 74.41 & 2.56 & 10 \\
      \texttt{PageRankSampler(alpha=0.85)} & & & 73.53 & 2.28 & 11 \\
      \texttt{PageRankSampler(split=True,alpha=0.85)} & & & 71.18 & 2.39 & 12 \\
      \texttt{ObjPredFreqSampler} & & & 68.82 & 3.53 & 13 \\
      \midrule
      \texttt{ObjPredFreqSampler(inverse=True)} & \multirow{13}{*}{4} & \multirow{13}{*}{500} & \textbf{77.94} & 0.93 & 1 \\
      \texttt{ObjFreqSampler(inverse=True)} & & & 77.65 & 1.95 & 2 \\
      \texttt{UniformSampler} & & & 77.35 & 1.18 & 3 \\
      \texttt{PageRankSampler(split=True,alpha=0.85)} & & & 77.35 & 2.73 & 4 \\
      \texttt{ObjPredFreqSampler} & & & 77.06 & 2.39 & 5 \\
      \texttt{ObjFreqSampler(inverse=True,split=True)} & & & 76.18 & 1.10 & 6 \\
      \texttt{PredFreqSampler} & & & 75.59 & 4.32 & 7 \\
      \texttt{PageRankSampler(alpha=0.85)} & & & 75.59 & 3.03 & 8 \\
      \texttt{PageRankSampler(inverse=True,split=True,alpha=0.85)} & & & 75.29 & 2.16 & 9 \\
      \texttt{PageRankSampler(inverse=True,alpha=0.85)} & & & 75.29 & 3.14 & 10 \\
      \texttt{WideSampler} & & & 74.41 & 2.20 & 11 \\
      \texttt{ObjFreqSampler} & & & 74.41 & 2.20 & 11 \\
      \texttt{PredFreqSampler(inverse=True)} & & & 74.12 & 4.12 & 12 \\
      \bottomrule
    \end{tabular}
    }%
    \caption{Accuracy of Sampling Strategies for \texttt{MUTAG} (Part II).}
  \label{tab:benchmarks:samplers:part:2}
\end{table}

In Table \ref{tab:benchmarks:samplers:part:2}, for a maximum depth per walk of 4
with a maximum number of walks of 100, \texttt{WideSampler} indicates the best
model's accuracy. However, the model's accuracy of \texttt{WideSampler} is
inversely proportional to the maximum number of walks per entity.

\newpage

 \begin{figure}[!ht]
  \centering
  \resizebox{\textwidth}{!}{%
    \begin{tikzpicture}
      \begin{axis}[
        scale only axis,
        grid=major,
        grid style={dashed,gray!30},
        height=6cm,
        width=9cm,
        legend cell align={left},
        legend entries={
          \texttt{ObjFreqSampler(inverse=True,split=True)},
          \texttt{ObjFreqSampler(inverse=True)},
          \texttt{ObjFreqSampler},
          \texttt{ObjPredFreqSampler(inverse=True)},
          \texttt{ObjPredFreqSampler},
          \texttt{PageRankSampler(inverse=True,split=True,alpha=0.85)},
          \texttt{PageRankSampler(inverse=True,alpha=0.85)},
          \texttt{PageRankSampler(split=True,alpha=0.85)},
          \texttt{PageRankSampler(alpha=0.85)},
          \texttt{PredFreqSampler(inverse=True)},
          \texttt{PredFreqSampler},
          \texttt{UniformSampler},
          \texttt{WideSampler},
        },
        legend style={
          legend pos=outer north east,
          font=\small
        },
        ylabel={Accuracy},
        xlabel={Maximum Number of Walks per Entity},
        xtick={100,250,500},
        ytick={70,72,74,76,78}
        ]

        \addplot[medimumBlue,mark=diamond,error bars/.cd, y dir=both, y explicit]
        table[x=max_walk,y=accuracy,col sep=comma] {data/samplers/max-depth/4/objfreq-inv-split.csv};
        \addplot[darkBlue,mark=square,error bars/.cd, y dir=both, y explicit]
        table[x=max_walk,y=accuracy,col sep=comma] {data/samplers/max-depth/4/objfreq-inv.csv};
        \addplot[blue,mark=*,error bars/.cd, y dir=both, y explicit]
        table[x=max_walk,y=accuracy,col sep=comma] {data/samplers/max-depth/4/objfreq.csv};

        \addplot[darkRed,mark=square,error bars/.cd, y dir=both, y explicit]
        table[x=max_walk,y=accuracy,col sep=comma] {data/samplers/max-depth/4/objpredfreq-inv.csv};
        \addplot[red,mark=*,error bars/.cd, y dir=both, y explicit]
        table[x=max_walk,y=accuracy,col sep=comma] {data/samplers/max-depth/4/objpredfreq.csv};

        \addplot[mediumGreen,mark=diamond,error bars/.cd, y dir=both, y explicit]
        table[x=max_walk,y=accuracy,col sep=comma] {data/samplers/max-depth/4/pagerank-inv-split.csv};
        \addplot[darkGreen,mark=square,error bars/.cd, y dir=both, y explicit]
        table[x=max_walk,y=accuracy,col sep=comma] {data/samplers/max-depth/4/pagerank-inv.csv};
        \addplot[yellow,mark=triangle,error bars/.cd, y dir=both, y explicit]
        table[x=max_walk,y=accuracy,col sep=comma]{data/samplers/max-depth/4/pagerank-split.csv};
        \addplot[green,mark=*,error bars/.cd, y dir=both, y explicit]
        table[x=max_walk,y=accuracy,col sep=comma] {data/samplers/max-depth/4/pagerank.csv};

        \addplot[darkPurple,mark=square,error bars/.cd, y dir=both, y explicit]
        table[x=max_walk,y=accuracy,col sep=comma] {data/samplers/max-depth/4/predfreq-inv.csv};
        \addplot[purple,mark=*,error bars/.cd, y dir=both, y explicit]
        table[x=max_walk,y=accuracy,col sep=comma] {data/samplers/max-depth/4/predfreq.csv};

        \addplot[brown,mark=*,error bars/.cd, y dir=both, y explicit]
        table[x=max_walk,y=accuracy,col sep=comma] {data/samplers/max-depth/4/uniform.csv};
        \addplot[black,mark=*,error bars/.cd, y dir=both, y explicit]
        table[x=max_walk,y=accuracy,col sep=comma] {data/samplers/max-depth/4/widesampler.csv};
      \end{axis}
    \end{tikzpicture}
  }%
  \caption{Sampling Strategies for \texttt{MUTAG} According
    to a Maximum Depth per Walk of 4.}
  \label{fig:benchmarks:samplers:part:2}
\end{figure}

In Figure \ref{fig:benchmarks:samplers:part:2}, the model's accuracy for most of
the walking strategies is proportional to the maximum number of walks per
entity.

\begin{table}[!ht]
  \centering
  \resizebox{0.85\textwidth}{!}{%
    \begin{tabular}{lcc S[table-format=2.2]@{${}\pm{}$}S[table-format=1.2]c}
      \toprule
      \textbf{Sampler} & \textbf{Max. Depth} & \textbf{Max. Walks}
      & \multicolumn{2}{c}{\textbf{Accuracy} (\SI{}{\percent})} & \textbf{Rank} \\
      \midrule
      \texttt{PredFreqSampler} & \multirow{13}{*}{6} & \multirow{13}{*}{100} & \textbf{79.41} & 2.46 & 1 \\
      \texttt{PageRankSampler(split=True,alpha=0.85)} & & & 77.06 & 3.03 & 2 \\
      \texttt{ObjFreqSampler(inverse=True)} & & & 76.76 & 3.40 & 3 \\
      \texttt{WideSampler} & & & 76.76 & 3.40 & 3 \\
      \texttt{PageRankSampler(inverse=True,alpha=0.85)} & & & 76.76 & 3.88 & 4 \\
      \texttt{ObjFreqSampler} & & & 75.59 & 5.06 & 5 \\
      \texttt{ObjPredFreqSampler(inverse=True)} & & & 75.59 & 2.73 & 6 \\
      \texttt{PageRankSampler(inverse=True,split=True,alpha=0.85)} & & & 75.29 & 1.71 & 7 \\
      \texttt{PredFreqSampler(inverse=True)} & & & 75.29 & 1.71 & 8 \\
      \texttt{ObjFreqSampler(inverse=True,split=True)} & & & 74.12 & 3.03 & 9 \\
      \texttt{PageRankSampler(alpha=0.85)} & & & 73.53 & 3.35 & 10 \\
      \texttt{UniformSampler} & & & 73.24 & 3.77 & 11 \\
      \texttt{ObjPredFreqSampler} & & & 70.88 & 6.47 & 12 \\
      \midrule
      \texttt{PageRankSampler(inverse=True,alpha=0.85)} & \multirow{13}{*}{6} & \multirow{13}{*}{250} & \textbf{80.00} & 1.50 & 1 \\
      \texttt{ObjFreqSampler(inverse=True,split=True)} & & & 78.53 & 3.17 & 2 \\
      \texttt{PredFreqSampler(inverse=True)} & & & 78.24 & 4.30 & 3 \\
      \texttt{ObjPredFreqSampler(inverse=True)} & & & 78.24 & 6.20 & 4 \\
      \texttt{PageRankSampler(alpha=0.85)} & & & 77.35 & 2.39 & 5 \\
      \texttt{PageRankSampler(inverse=True,split=True,alpha=0.85)} & & & 76.18 & 2.53 & 6 \\
      \texttt{ObjFreqSampler(inverse=True)} & & & 76.18 & 4.30 & 7 \\
      \texttt{UniformSampler} & & & 75.59 & 1.76 & 8 \\
      \texttt{PredFreqSampler} & & & 75.59 & 3.03 & 9 \\
      \texttt{ObjFreqSampler} & & & 75.00 & 2.08 & 10 \\
      \texttt{WideSampler} & & & 75.00 & 4.83 & 11 \\
      \texttt{ObjPredFreqSampler} & & & 73.24 & 4.78 & 12 \\
      \texttt{PageRankSampler(split=True,alpha=0.85)} & & & 72.65 & 2.56 & 13 \\
      \midrule
      \texttt{PredFreqSampler} & \multirow{13}{*}{6} & \multirow{13}{*}{500} & \textbf{79.41} & 3.08 & 1 \\
      \texttt{ObjFreqSampler(inverse=True)} & & & 78.82 & 1.50 & 2 \\
      \texttt{WideSampler} & & & 78.53 & 1.18 & 3 \\
      \texttt{ObjFreqSampler} & & & 78.24 & 1.95 & 4 \\
      \texttt{PageRankSampler(inverse=True,split=True,alpha=0.85)} & & & 77.94 & 1.32 & 5 \\
      \texttt{UniformSampler} & & & 77.94 & 1.86 & 6 \\
      \texttt{PredFreqSampler(inverse=True)} & & & 77.35 & 1.50 & 7 \\
      \texttt{PageRankSampler(alpha=0.85)} & & & 77.35 & 1.50 & 7 \\
      \texttt{PageRankSampler(inverse=True,alpha=0.85)} & & & 77.35 & 1.99 & 8 \\
      \texttt{ObjPredFreqSampler} & & & 77.06 & 1.99 & 9 \\
      \texttt{PageRankSampler(split=True,alpha=0.85)} & & & 76.76 & 2.16 & 10 \\
      \texttt{ObjFreqSampler(inverse=True,split=True)} & & & 75.88 & 1.76 & 11 \\
      \texttt{ObjPredFreqSampler(inverse=True)} & & & 75.59 & 3.55 & 12 \\
      \bottomrule
    \end{tabular}
    }%
    \caption{Accuracy of Sampling Strategies for MUTAG (Part III).}
    \label{tab:benchmark:samplers:walks:part:3}
\end{table}

In Table \ref{tab:benchmark:samplers:walks:part:3}, for a maximum depth of walk of 6 with
a maximum number of walks of 500, \texttt{WideSampler} indicates a good
model's accuracy. However, the latter indicates a poor model's accuracy for a
maximum number of walks of 500.

\newpage

\begin{figure}[!ht]
  \centering
  \resizebox{\textwidth}{!}{%
    \begin{tikzpicture}
      \begin{axis}[
        scale only axis,
        grid=major,
        grid style={dashed,gray!30},
        height=6cm,
        width=9cm,
        legend cell align={left},
        legend entries={
          \texttt{ObjFreqSampler(inverse=True,split=True)},
          \texttt{ObjFreqSampler(inverse=True)},
          \texttt{ObjFreqSampler},
          \texttt{ObjPredFreqSampler(inverse=True)},
          \texttt{ObjPredFreqSampler},
          \texttt{PageRankSampler(inverse=True,split=True,alpha=0.85)},
          \texttt{PageRankSampler(inverse=True,alpha=0.85)},
          \texttt{PageRankSampler(split=True,alpha=0.85)},
          \texttt{PageRankSampler(alpha=0.85)},
          \texttt{PredFreqSampler(inverse=True)},
          \texttt{PredFreqSampler},
          \texttt{UniformSampler},
          \texttt{WideSampler},
        },
        legend style={
          legend pos=outer north east,
          font=\small
        },
        ylabel={Accuracy},
        xlabel={Maximum Number of Walks per Entity},
        xtick={100,250,500},
        ytick={72,74,76,78,80}
        ]

        \addplot[medimumBlue,mark=diamond,error bars/.cd, y dir=both, y explicit]
        table[x=max_walk,y=accuracy,col sep=comma] {data/samplers/max-depth/6/objfreq-inv-split.csv};
        \addplot[darkBlue,mark=square,error bars/.cd, y dir=both, y explicit]
        table[x=max_walk,y=accuracy,col sep=comma] {data/samplers/max-depth/6/objfreq-inv.csv};
        \addplot[blue,mark=*,error bars/.cd, y dir=both, y explicit]
        table[x=max_walk,y=accuracy,col sep=comma] {data/samplers/max-depth/6/objfreq.csv};

        \addplot[darkRed,mark=square,error bars/.cd, y dir=both, y explicit]
        table[x=max_walk,y=accuracy,col sep=comma] {data/samplers/max-depth/6/objpredfreq-inv.csv};
        \addplot[red,mark=*,error bars/.cd, y dir=both, y explicit]
        table[x=max_walk,y=accuracy,col sep=comma] {data/samplers/max-depth/6/objpredfreq.csv};

        \addplot[mediumGreen,mark=diamond,error bars/.cd, y dir=both, y explicit]
        table[x=max_walk,y=accuracy,col sep=comma] {data/samplers/max-depth/6/pagerank-inv-split.csv};
        \addplot[darkGreen,mark=square,error bars/.cd, y dir=both, y explicit]
        table[x=max_walk,y=accuracy,col sep=comma] {data/samplers/max-depth/6/pagerank-inv.csv};
        \addplot[yellow,mark=triangle,error bars/.cd, y dir=both, y explicit]
        table[x=max_walk,y=accuracy,col sep=comma]{data/samplers/max-depth/6/pagerank-split.csv};
        \addplot[green,mark=*,error bars/.cd, y dir=both, y explicit]
        table[x=max_walk,y=accuracy,col sep=comma] {data/samplers/max-depth/6/pagerank.csv};

        \addplot[darkPurple,mark=square,error bars/.cd, y dir=both, y explicit]
        table[x=max_walk,y=accuracy,col sep=comma] {data/samplers/max-depth/6/predfreq-inv.csv};
        \addplot[purple,mark=*,error bars/.cd, y dir=both, y explicit]
        table[x=max_walk,y=accuracy,col sep=comma] {data/samplers/max-depth/6/predfreq.csv};

        \addplot[brown,mark=*,error bars/.cd, y dir=both, y explicit]
        table[x=max_walk,y=accuracy,col sep=comma] {data/samplers/max-depth/6/uniform.csv};
        \addplot[black,mark=*,error bars/.cd, y dir=both, y explicit]
        table[x=max_walk,y=accuracy,col sep=comma] {data/samplers/max-depth/6/widesampler.csv};
      \end{axis}
    \end{tikzpicture}
  }%
  \caption{Sampling Strategies for \texttt{MUTAG} According
    to a Maximum Depth per Walk of 6.}
  \label{fig:benchmarks:samplers:part:3}
\end{figure}

In Figure \ref{fig:benchmarks:samplers:part:3}, the curve of
\texttt{WideSampler} shows the same trajectory as \texttt{PredFreqSampler} and
\texttt{PageRankSampler(inverse=True,alpha=0.85)}. In addition,
\texttt{ObjPreqFreqSampler} is proportional to the maximum number of walks per
entity.

\begin{table}[!ht]
  \centering
  \begin{tabular}{lc}
    \toprule
    \textbf{Sampler} & \textbf{Average Rank} \\
    \midrule
    \texttt{WideSampler} & 1 \\
    \texttt{ObjPredFreqSampler(inverse=True)} & 2 \\
    \texttt{PredFreqSampler} & 3 \\
    \texttt{ObjFreqSampler} & 4 \\
    \texttt{ObjFreqSampler(inverse=True)} & 5 \\
    \texttt{PageRankSampler(inverse=True,alpha=0.85)} & 6 \\
    \texttt{PageRankSampler(inverse=True,split=True,alpha=0.85)} & 7 \\
    \texttt{ObjFreqSampler(inverse=True,split=True)} & 8 \\
    \texttt{PageRankSampler(split=True,alpha=0.85)} & 9 \\
    \texttt{PredFreqSampler(inverse=True)} & 10 \\
    \texttt{ObjPredFreqSampler} & 11 \\
    \texttt{UniformSampler} & 11 \\
    \texttt{PageRankSampler(alpha=0.85)} & 12 \\
    \bottomrule
  \end{tabular}
  \caption{Average Rank of the Sampling Strategies.}
  \label{benchmark:samplers:average:rank}
\end{table}

In Figure \ref{benchmark:samplers:average:rank}, \texttt{WideSampler} is the winning
sampler strategy in these benchmarks for \texttt{MUTAG}, followed by
\texttt{ObjPredFreqSampler(inverse=True)}, and \texttt{PredFreqSampler}.

%%% Local Variables:
%%% mode: latex
%%% TeX-master: "../../master-thesis"
%%% End:


%%% Local Variables:
%%% mode: latex
%%% TeX-master: "../master-thesis"
%%% End: