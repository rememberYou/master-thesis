
\chapter{Objectives}
\label{chap:objectives}

This chapter covers the specifications expected by this Master's thesis and the
problems faced in its elaboration. In addition, each specification is followed
by a summary of what was done to get a better idea of the overall report.

\section{Specifications}
\label{sec:objectives:specifications}

In 2018, BERT was released and outperformed all other existing techniques on
various ML tasks related to various fields. This Master's thesis aims to
integrate BERT into
\texttt{pyRDF2Vec}\footnote{\url{https://github.com/IBCNServices/pyRDF2Vec/}},
but more specifically to evaluate its impact in terms of runtime, predictive
performance, and memory usage using different combinations of walking and
sampling strategies. Moreover, by reading literature from general graph-based ML
and NLP research, inspiration can be found to design new sampling and walking
strategies that result in improved predictive performances by exploiting the
properties of BERT. The specifications of this Master's thesis are composed of
the following four points:
\begin{enumerate}
\item \textbf{Support BERT and other embedding techniques for comparison
purposes}: at least the BERT embedding technique should be integrated into
\texttt{pyRDF2Vec} and be compared to the Word2Vec technique, which is currently
used. Additionally, the extension of Word2Vec, namely FastText, will also have
to be implemented and be compared to know its impact.

For this Master's thesis, BERT and FastText have been implemented and integrated
within the \texttt{pyRDF2Vec} library. Added to this implementation is a new
architecture for \texttt{pyRDF2Vec} that easily allows other embedding
techniques. Since Word2Vec already exists in the library, this Master thesis
proposed a better walk extraction to improve the model's accuracy.

\item \textbf{Evaluate the impact of BERT}: A thorough benchmarking study to
evaluate the impact of the BERT embedding technique on different dimensions will
have to be conducted. For this, data sets having different properties and
stemming from different domains will have to be selected. Moreover, a rigorous
framework that performs the required experiments and logs all of these results
will have to be developed.

For this Master's thesis, BERT, Word2Vec, and FastText are compared on
\texttt{MUTAG}, a small KG. In addition to these benchmarks of embedding
techniques, the \texttt{pyRDF2Vec} library was modernized, allowing to get more
information with the \texttt{verbose} parameter of the
\texttt{RDF2VecTransformer} class.

\newpage

\item \textbf{Support of new walking strategies for RDF2Vec}: while an initial
set of five simple walking strategies are already implemented in
\texttt{pyRDF2Vec}, many other alternatives exist. Implementing more walking
strategies would allow more extensive and detailed comparisons for embedding
techniques.

For this Master's thesis, the \texttt{SplitWalker} walking strategy is proposed and
implicitly introduced some of its variants. Finally, this Master's thesis offers
some benchmarks related to \texttt{SplitWalker} and other existing walking strategies.

\item \textbf{Support of new sampling strategies for RDF2Vec}: similarly, other
sampling strategies can be created and implemented in
\texttt{pyRDF2Vec}. Therefore, it will be possible to see the impact of the
choice of a walking strategy and a sampling strategy with the performances
related to BERT and other embeddings techniques.

For this Master's thesis, the \texttt{WideSampler} walking strategy is proposed,
leading to other variants of the latter. In addition, this Master's thesis
evaluates \texttt{WideSampler} with other existing sampling strategies to knows
its impact.
\end{enumerate}

\section{Problems Encountered}
\label{sec:objectives:problems}

This section discusses the problems that have been encountered and why some
benchmarks are missing. Initially, benchmarks related to bigger KGs such as
\texttt{AM} and \texttt{DBP:Cities} should have been done. Therefore, it is
helpful to understand the reasons for this to prevent these errors from
recurring.

\subsection{Garbage Collector}
\label{subsec:garbage:collector}

This Master's thesis being coupled with the internship, the same servers were
used. Due to an internal problem with IDLab's server infrastructure, a loss of
time equivalent to one and a half months of work was made. Therefore, this loss
of time had repercussions on this Master's thesis. Specifically, the benchmarks
related to BERT are not presented for this version of the report. However, they
will be added for the next version. Moreover, some benchmarks had to be
shortened due to time constraints.

For more information related to this issue, IDLab servers use the
Stardog\footnote{\url{https://www.stardog.com/}} infrastructure, which is
implemented in Java. Therefore, the different benchmarks for the internship and
Master's thesis recorded high variants. After several weeks of debugging,
thinking that this was a \texttt{pyRDF2Vec} issue, a Stardog engineer confirmed
that the issue was related to their infrastructure. In more detail, the concern
was the lack of RAM release from the servers due to garbage collection. As a
result, new data was being saved directly to a physical disk, which resulted in
much higher latencies than RAM. From then on, this was reflected in the variance
of the results.

\subsection{Exceptions Raised}
\label{subsec:exception:raised}

In addition to the problems related to the garbage collector, exceptions are
encountered from time to time during SPARQL Protocol and RDF Query Language
(SPARQL) queries. These exceptions are due to the inaccessible locally hosted
SPARQL endpoint during updates or internal network problems. As a result, after
more than three attempts to retrieve the walks of a provided entity, an
exception is thrown, which makes the benchmarks stop. Since these benchmarks can
take several days, it also delayed other benchmarks.

%%% Local Variables:
%%% mode: latex
%%% TeX-master: "latex"
%%% End:
