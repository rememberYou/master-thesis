
\section{Graphs}
\label{sec:background:graphs}

Graphs have been used long before the introduction of ML and cover a wide range
of applications. To better understand the precise vocabulary of graphs, this
section provides some definitions.

\begin{definition}[Graph]
  Ordered pair ($V, E$), where $V$ is a finite and non-empty set of elements
  called \emph{vertices} (or \emph{nodes}), and $E$ is a set of unordered pairs of
  distinct nodes of $V$, called \emph{edges}.
  \label{def:graph}
\end{definition}

\noindent Basic types of graphs include:
\begin{figure}[!ht]
  \begin{minipage}{0.45\linewidth}
    \centering
    \begin{tikzpicture}[minimum size=0.5cm,node distance=1cm]
      \node[entity] (alpha) {1};
      \node[entity,above left=of alpha,xshift=15pt] (beta) {2};
      \node[entity,above right=of alpha,yshift=-5pt,xshift=-5pt] (gamma) {3};
      \node[entity,right=of gamma] (delta) {4};

      \draw[arrow] (alpha) -- (beta);
      \draw[arrow] (beta) -- (gamma);
      \draw[arrow] (alpha) -- (gamma);
      \draw[arrow] (gamma) -- (delta);
    \end{tikzpicture}
  \end{minipage}
  %
  \begin{minipage}{0.5\linewidth}
    \begin{definition}[Oriented Graph]
      Directed Graph where bidirected edges connect no pair of vertices.
      \label{def:oriented:graph}
    \end{definition}
  \end{minipage}
  \vfill
  \vspace{\belowdisplayskip}
  \begin{minipage}{0.45\linewidth}
    \centering
    \begin{tikzpicture}[minimum size=0.5cm,node distance=1cm]
      \node[entity] (alpha) {1};
      \node[entity,above left=of alpha,xshift=15pt] (beta) {2};
      \node[entity,above right=of alpha,yshift=-5pt,xshift=-5pt] (gamma) {3};
      \node[entity,right=of gamma] (delta) {4};

      \draw[arrow] (alpha) -- (beta);
      \draw[arrow] (beta) -- (gamma);
      \draw[arrow] (alpha) -- (gamma);

      \path[arrow] let \p1=($(gamma)-(delta)$),\n1={atan2(\y1,\x1)},\n2={180+\n1} in
      ($ (delta.\n1)!2pt!90:(gamma.\n2) $) edge node {} ($ (gamma.\n2)!2pt!-90:(delta.\n1) $);
      \path[arrow] let \p1=($(delta)-(gamma)$),\n1={atan2(\y1,\x1)},\n2={180+\n1} in
      ($ (gamma.\n1)!2pt!90:(delta.\n2) $) edge node {} ($ (delta.\n2)!2pt!-90:(gamma.\n1)
      $);
    \end{tikzpicture}
  \end{minipage}
  %
  \begin{minipage}{0.5\linewidth}
    \begin{definition}[Directed Graph]
      Named \emph{digraph}, it is defined as a graph whose edges have an
      orientation, also known as \emph{directed edges}, \emph{directed links},
      \emph{arrows}, or \emph{arcs}.
      \label{def:directed:graph}
    \end{definition}
  \end{minipage}
  \caption{Basic Graph Types (Part I).}
\end{figure}

\newpage

\begin{figure}[!ht]
  \begin{minipage}{0.45\linewidth}
    \centering
    \begin{tikzpicture}[minimum size=0.5cm,node distance=1cm]
      \node[entity] (alpha) {1};
      \node[entity,above left=of alpha,xshift=15pt] (beta) {2};
      \node[entity,above right=of alpha,yshift=-5pt,xshift=-5pt] (gamma) {3};
      \node[entity,right=of gamma] (delta) {4};

      \draw (alpha) -- (beta);
      \draw (beta) -- (gamma);
      \draw (alpha) -- (gamma);
      \draw (gamma) -- (delta);
    \end{tikzpicture}
  \end{minipage}
  %
  \begin{minipage}{0.5\linewidth}
    \begin{definition}[Undirected Graph]
      Graph whose edges are bidirectional.
      \label{def:undirected:graph}
    \end{definition}
  \end{minipage}
  \vfill
  \vspace{\belowdisplayskip}
  \begin{minipage}{0.45\linewidth}
    \centering
    \begin{tikzpicture}[minimum size=0.5cm,node distance=1cm]
      \node[entity] (alpha) {1};
      \node[entity,above left=of alpha,xshift=15pt] (beta) {2};
      \node[entity,above right=of alpha,yshift=-5pt,xshift=-5pt] (gamma) {3};
      \node[entity,right=of gamma] (delta) {4};

      \draw[thick] (alpha) -- (beta);
      \draw[thick] (beta) -- (gamma);
      \draw[thick] (alpha) -- (gamma);

      \path[thick] let \p1=($(gamma)-(delta)$),\n1={atan2(\y1,\x1)},\n2={180+\n1} in
      ($ (delta.\n1)!2pt!90:(gamma.\n2) $) edge node {} ($ (gamma.\n2)!2pt!-90:(delta.\n1) $);
      \path[thick] let \p1=($(delta)-(gamma)$),\n1={atan2(\y1,\x1)},\n2={180+\n1} in
      ($ (gamma.\n1)!2pt!90:(delta.\n2) $) edge node {} ($ (delta.\n2)!2pt!-90:(gamma.\n1)
      $);
    \end{tikzpicture}
  \end{minipage}
  %
  \begin{minipage}{0.5\linewidth}
    \begin{definition}[Multigraph]
      Undirected graph that can store multiple edges between two nodes.
      \label{def:multigraph}
    \end{definition}
  \end{minipage}
  \caption{Basic Graph Types (Part II).}
\end{figure}

\noindent Based on these definitions of graph types, the KG is defined.
\begin{definition}[Knowledge Graph]
Directed \emph{heterogeneous} multigraph whose node and relation types have
domain-specific semantics~\citep{website:kamakoti}. These nodes can be of
different types. From a terminology point of view, the nodes/vertices of a KG
are often called \emph{entities}, and the directed edges refer to as
predicates. Moreover, this multigraph has \emph{triple}\footnote{Also called
\emph{triplets}.} designing a 3-tuple, where each triple defines a
(\texttt{subject}, \texttt{predicate}, \texttt{object}) tuple\footnote{Some
authors define this tuple as \texttt{(h, r, t)}. In this definition, \texttt{h}
is the head entity, \texttt{t} the tail entity, and \texttt{r} the relation
associating the head with the tail entities.}. For ease of processing, the
multigraph aspect of the KG can be removed by representing each triple as two
2-tuple: (\texttt{subject} $\rightarrow$ \texttt{predicate}) and
(\texttt{predicate} $\rightarrow$ \texttt{object}).
\end{definition}

%%% Local Variables:
%%% mode: latex
%%% TeX-master: "../../report"
%%% End:
